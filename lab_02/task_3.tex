\documentclass{article}
\usepackage{sagetex}
\setlength{\sagetexindent}{10ex}
\usepackage{amsmath}
\usepackage[russian]{babel}
\usepackage{hyperref}
\usepackage[left=15mm,top=30mm,bottom=30mm,right=15mm]{geometry}
\begin{document}
\begin{center}
\Large{\textbf{Лабораторная работа 2. Задание 3 - приведение поверхности второго порядка к каноническому виду.}}
\end{center}


\section*{Задание}
\begin{enumerate}
\item Упростить уравнение поверхности второго порядка в пространстве;
\item Привести к каноническому виду;
\item Построить исходную фигуру и упрощенную;
\item Сравнить вручную найденные собственные числа и собственные вектора со значениями, полученными с помощью встроенных функцийю
\end{enumerate}

\begin{center}
Вариант 7

$8x^2 - 2xy - 4y^2 + 2xz - 2yz + 3z^2 + 7x + 8y + 9z - 10$
\end{center}

\section*{Построение исходной поверхности}
Функция:

\begin{sageblock}
f(x, y, z) = 8*x**2 - 2*x*y - 4*y**2 + 2*x*z - 2*y*z + 3*z**2 + 7*x + 8*y + 9*z - 10
\end{sageblock}
Выведем ее:
\begin{center}
$\sage{f}$
\end{center}
Построим ее график:
\begin{center}
\sageplot{implicit_plot3d(f, (x, -30, 30), (y, -30, 30), (z, -30, 30), color = 'green', figsize = 3)}
\end{center}
\newpage
\section*{Приведем поверхность к каноническому виду}

Общее уравнение поверхности второго порядка имеет вид

$Ax^2 + By^2 + Cz^2 + 2Fxy + 2Gyz + 2Hzx + 2Px + 2Qy + 2Rz + D = 0$.

Составим матрицу квадратичной формы A и матрицу квадратичной, линейной форм и свободного члена K:
\begin{center}

$A =
\begin{pmatrix}
A & F & H\\
F & B & G\\
H & G & C
\end{pmatrix}\
K =
\begin{pmatrix}
A & F & H & P\\
F & B & G & Q\\
H & G & C & R\\
P & Q & R & D
\end{pmatrix}$
\end{center}

\begin{sageblock}
A = matrix([[8, -1, 1],
          [-1, -4, -1],
          [1, -1, 3]])

K = matrix([[8, -1, 1, 3.5],
           [-1, -4, -1, 4],
           [1, -1, 3, 4.5],
           [3.5, 4, 4.5, -10]])
\end{sageblock}

Найдем инварианты:

$I_1 = A+B+C$

$I_2 = \begin{pmatrix}
A & F\\
F & B \end{pmatrix}\
+ \begin{pmatrix}
A & H\\
H & C \end{pmatrix}\
+ \begin{pmatrix}
B & G\\
G & C \end{pmatrix}$

$I_3 = det(A)$

$I_4 = det(K)/I_3$

$K_1 = det(k)$

\begin{sageblock}
I1 = A.trace()
I2 = A[0:2, 0:2].det() + A[[0, 2], [0, 2]].det() + A[1:3, 1:3].det()
I3 = det(A)
I4 = K.det()/I3
K1 = K.det()
\end{sageblock}

Получили следующие результаты:

\begin{center}
$I_1 = \sage{I1}$

$I_2 = \sage{I2}$

$I_3 = \sage{I3}$

$I_4 = \sage{I4}$

$K_1 = \sage{K1}$

\end{center}

Сравним полученные инварианты $I_1, I_2, I_3$ с инвариантами характеристического уровнения $\lambda{^3} - I_1\lambda{^2}+I_2\lambda-I_3=0$, полеченного с помощью встроенной функции .charpoly()

$\sage{A.charpoly(x)}$

Они совпадают, значит инварианты получены верно. Теперь найдем корни характеристического уравнения:

\begin{sagesilent}
var("v")
\end{sagesilent}

\begin{sageblock}
E = matrix([
    [1, 0, 0],
    [0, 1, 0],
    [0, 0, 1]
])

values = []
for value in solve((A - v * E).det(), v):
    values.append(real_part(n(value.rhs())))
\end{sageblock}

\begin{center}
$\lambda_1 = \sage{values[0]}$

$\lambda_2 = \sage{values[1]}$

$\lambda_3 = \sage{values[2]}$

\end{center}

Сравним полученные корни характеристического уравнения с корнями, полученными с помощью встроенной функции .eigenvalues():

$\sage{A.eigenvalues()}$

Совпадают. Значит теперь на основе корней характеристического уравнения и инвариант можно понять, какая перед нами поверхность. Так как два корня характеристического уравнения имеют один знак, а третий корень и $I_4$ имеют знак, им противоположный, то общее уравнение поверхности второго порядка определяет однополостный гиперболоид.

Найдем a, b, c:

$a = \sqrt{\frac{-K_1}{I_3\lambda_1}}, $
$b = \sqrt{\frac{-K_1}{I_3\lambda_2}}, $
$c = \sqrt{\frac{-K_1}{I_3\lambda_3}}$

\begin{sageblock}
a = -sqrt(K1/(I3*values[0]))
b = -sqrt(K1/(I3*values[1]))
c = -sqrt(K1/(I3*values[2]))
\end{sageblock}

Каноническое уравнение имеет вид $\frac{x^2}{a^2}+\frac{y^2}{b^2}-\frac{z^2}{c^2} - 1$.

\begin{sageblock}
f_canon(x1, y1, z1) = x1**2 / a**2 + y1**2 / b**2 - z1**2 / c**2 - 1
\end{sageblock}

$f = \sage{f_canon(x1, y1, z1)}$

Построим график канонической функции:

\begin{center}
\sageplot{implicit_plot3d(f_canon, (x1, -30, 30), (y1, -30, 30), (z1, -30, 30), color = 'red', figsize = 3)}
\end{center}
\end{document}
